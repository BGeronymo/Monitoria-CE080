\documentclass[]{book}
\usepackage{lmodern}
\usepackage{amssymb,amsmath}
\usepackage{ifxetex,ifluatex}
\usepackage{fixltx2e} % provides \textsubscript
\ifnum 0\ifxetex 1\fi\ifluatex 1\fi=0 % if pdftex
  \usepackage[T1]{fontenc}
  \usepackage[utf8]{inputenc}
\else % if luatex or xelatex
  \ifxetex
    \usepackage{mathspec}
  \else
    \usepackage{fontspec}
  \fi
  \defaultfontfeatures{Ligatures=TeX,Scale=MatchLowercase}
\fi
% use upquote if available, for straight quotes in verbatim environments
\IfFileExists{upquote.sty}{\usepackage{upquote}}{}
% use microtype if available
\IfFileExists{microtype.sty}{%
\usepackage{microtype}
\UseMicrotypeSet[protrusion]{basicmath} % disable protrusion for tt fonts
}{}
\usepackage[margin=1in]{geometry}
\usepackage{hyperref}
\hypersetup{unicode=true,
            pdftitle={CE080 - FUNDAMENTOS BÁSICOS PARA ESTATÍSTICA},
            pdfborder={0 0 0},
            breaklinks=true}
\urlstyle{same}  % don't use monospace font for urls
\usepackage{natbib}
\bibliographystyle{apalike}
\usepackage{longtable,booktabs}
\usepackage{graphicx,grffile}
\makeatletter
\def\maxwidth{\ifdim\Gin@nat@width>\linewidth\linewidth\else\Gin@nat@width\fi}
\def\maxheight{\ifdim\Gin@nat@height>\textheight\textheight\else\Gin@nat@height\fi}
\makeatother
% Scale images if necessary, so that they will not overflow the page
% margins by default, and it is still possible to overwrite the defaults
% using explicit options in \includegraphics[width, height, ...]{}
\setkeys{Gin}{width=\maxwidth,height=\maxheight,keepaspectratio}
\IfFileExists{parskip.sty}{%
\usepackage{parskip}
}{% else
\setlength{\parindent}{0pt}
\setlength{\parskip}{6pt plus 2pt minus 1pt}
}
\setlength{\emergencystretch}{3em}  % prevent overfull lines
\providecommand{\tightlist}{%
  \setlength{\itemsep}{0pt}\setlength{\parskip}{0pt}}
\setcounter{secnumdepth}{5}
% Redefines (sub)paragraphs to behave more like sections
\ifx\paragraph\undefined\else
\let\oldparagraph\paragraph
\renewcommand{\paragraph}[1]{\oldparagraph{#1}\mbox{}}
\fi
\ifx\subparagraph\undefined\else
\let\oldsubparagraph\subparagraph
\renewcommand{\subparagraph}[1]{\oldsubparagraph{#1}\mbox{}}
\fi

%%% Use protect on footnotes to avoid problems with footnotes in titles
\let\rmarkdownfootnote\footnote%
\def\footnote{\protect\rmarkdownfootnote}

%%% Change title format to be more compact
\usepackage{titling}

% Create subtitle command for use in maketitle
\newcommand{\subtitle}[1]{
  \posttitle{
    \begin{center}\large#1\end{center}
    }
}

\setlength{\droptitle}{-2em}
  \title{CE080 - FUNDAMENTOS BÁSICOS PARA ESTATÍSTICA}
  \pretitle{\vspace{\droptitle}\centering\huge}
  \posttitle{\par}
  \author{}
  \preauthor{}\postauthor{}
  \predate{\centering\large\emph}
  \postdate{\par}
  \date{2018-03-18}

\usepackage{booktabs}
\usepackage{amsthm}
\makeatletter
\def\thm@space@setup{%
  \thm@preskip=8pt plus 2pt minus 4pt
  \thm@postskip=\thm@preskip
}
\makeatother
\usepackage[brazil]{babel}

\begin{document}
\maketitle

{
\setcounter{tocdepth}{1}
\tableofcontents
}
\chapter*{Prefácio}\label{prefacio}
\addcontentsline{toc}{chapter}{Prefácio}

Este material busca auxiliar na compreensão e nos estudos dos assuntos
tratados na matéria \emph{Fundamentos Básicos para Estatística} (CE080)
ministrada pela professora
\href{http://leg.ufpr.br/doku.php/pessoais:fernanda}{\emph{Fernanda
Buhrer Rizzato}}
(\href{mailto:fernandab@ufpr.br}{\nolinkurl{fernandab@ufpr.br}}) no
primeiro semestre de 2018 para os candidatos presentes na terceira fase
de seleção de acadêmicos do curso de Estatística da Universidade Federal
do Paraná (\href{http://www.ufpr.br/}{UFPR}).

\chapter{Conjuntos Numéricos}\label{conjuntos-numericos}

Um conjunto numérico pode ser definido como um agrupamento de elementos
numéricos que possuem alguma característica em comum. Por exemplo,
podemos definir o conjunto dos números pares positivos como:

\[P = \{ 2,\ 4,\ 6,\ 8,\ 10,\ 12,\ ...\}\]

Onde a característica em comum entre os elementos de \(P\) é a
satisfação dos requisitos: i) ser par; ii) ser positivo.

De forma geral podemos denotar um conjunto em função das caracetísticas
em comum de seus elementos. Seja \(C\) um conjunto de elementos \(e\)
com uma característica em comum \(a\) definimos:

\[C = \{ e \mid e \ \text{possui a característica} \ a\}\] Lê-se: \(C\)
é o conjunto dos elementos \(e\) tal que \(e\) possui a característica
\(a\).

Retomando o exemplo do conjunto \(P\), podemos escrevê-lo em função de
suas características (i e ii):

\[P = \{x \mid x = 2n \ \forall \ n \in \mathbb{N}^{*} \}\]

Repare que \(n\) pertence a um conjunto denotado por \(\mathbb{N}^{*}\).
O conjunto \(\mathbb{N}^{*}\) possui apenas elementos positivos e
inteiros (zero não está incluso pois não é positivo e sim neutro). Logo
os valores de \(x\) serão os valores de \(n\) multiplicados por \(2\),
desta forma todos os valores de \(x\) serão positivos e pares. Falaremos
mais sobre o conjunto \(\mathbb{N}\) nas próximas seções.

\section{Números Inteiros}\label{numeros-inteiros}

O conjunto dos números inteiros, denotado por \(\mathbb{Z}\), compreende
todos os números inteiros não positivos (inclui o zero) e os números
inteiros positivos. Desta forma temos:

\[\mathbb{Z} = \{ ...,\ -3,\ -2,\ -1,\ 0,\ 1,\ 2,\ 3,\ ... \}\]

Na notação de conjunto podemos incluir três operadores:
\(*,\ + \ \text{e} \ -\) que servem respectivamente para denotar a
ausência do elemento neutro (zero), presença somente de elementos não
negativos e a presença somente de elementos não positivos. Com esses
operadores obtemos diversas variações do conjunto \(\mathbb{Z}\), ou
seja, subconjuntos do conjunto \(\mathbb{Z}\), são elas:

\begin{enumerate}
\def\labelenumi{\arabic{enumi}.}
\item
  Números inteiros não-nulos

  \(\mathbb{Z}^{*} = \{...,\ -3,\ -2,\ -1,\ 1,\ 2,\ 3,\ ... \}\)
\item
  Números inteiros não negativos

  \(\mathbb{Z}_{+} = \{0,\ 1,\ 2,\ 3,\ ... \}\)
\item
  Números inteiros não positivos

  \(\mathbb{Z}_{-} = \{...,\ -3,\ -2,\ -1,\ 0 \}\)
\item
  Números inteiros positivos

  \(\mathbb{Z}_{+}^{*} = \{ 1,\ 2,\ 3,\ ... \}\)
\item
  Números inteiros negativos

  \(\mathbb{Z}_{-}^{*} = \{...,\ -3,\ -2,\ -1 \}\)
\end{enumerate}

\section{Números Naturais}\label{numeros-naturais}

O conjunto dos números naturais, denotado por \(\mathbb{N}\), compreende
todos os números inteiros não negativos (inclui o zero). Desta forma
temos:

\[\mathbb{N} = \{ 0,\ 1,\ 2,\ 3,\ ... \}\]

Repare que o conjunto \(\mathbb{N}\) é um subconjunto de \(\mathbb{Z}\),
ou seja, \(\mathbb{N} \subset \mathbb{Z}\). Note também que
\(\mathbb{N}\) é equivalente a \(\mathbb{Z}_{+}\).

Na notação dos conjuntos dos números naturais podemos definir somente o
operador \(*\) uma vez que, por natureza, ele só possui elementos não
negativos.

\begin{enumerate}
\def\labelenumi{\arabic{enumi}.}
\item
  Números naturais positivos

  \(\mathbb{N}^{*} = \{ 1,\ 2,\ 3,\ ... \}\)
\end{enumerate}

\section{Números Racionais}\label{numeros-racionais}

O conjunto dos números naturais, denotado por \(\mathbb{Q}\), compreende
todos os números da forma \(\frac{a}{b}\) onde \(a \in \mathbb{Z}\) e
\(b \in \mathbb{Z}^{*}\). Desta forma temos:

\[\mathbb{Q} = \{ \frac{a}{b} \mid \ a \in \mathbb{Z},\ b \in \mathbb{Z}^{*} \}\]

Repare que o conjunto \(\mathbb{Z}\) é subconjunto de \(\mathbb{Q}\) e
\(\mathbb{N}\) é subconjunto de \(\mathbb{Z}\), ou seja
\(\mathbb{N} \subset \mathbb{Z} \subset \mathbb{Q}\). Então podemos
dizer que todos número natural é um número inteiro, e todo numero
inteiro é um número racional, sendo assim todo número natural é também
racional.

Na notação dos números racionais estão definidos os três operadores.

\begin{enumerate}
\def\labelenumi{\arabic{enumi}.}
\item
  Números racionais não-nulos

  \(\mathbb{Q}^{*} = \{ \frac{a}{b} \mid \ a \in \mathbb{Z}^{*},\ b \in \mathbb{Z}^{*} \}\)
\item
  Números racionais não negativos

  \(\mathbb{Q}_{+} = \{ \frac{a}{b} \mid \ a \in \mathbb{Z}_{+},\ b \in \mathbb{Z}_{+}^{*} \}\)

  \(\mathbb{Q}_{+} = \{ \frac{a}{b} \mid \ a \in \mathbb{Z}_{-},\ b \in \mathbb{Z}_{-}^{*} \}\)
\item
  Números racionais não positivos

  \(\mathbb{Q}_{-} = \{ \frac{a}{b} \mid \ a \in \mathbb{Z}_{-},\ b \in \mathbb{Z}_{+}^{*} \}\)

  \(\mathbb{Q}_{-} = \{ \frac{a}{b} \mid \ a \in \mathbb{Z}_{+},\ b \in \mathbb{Z}_{-}^{*} \}\)
\item
  Números racionais positivos

  \(\mathbb{Q}_{+}^{*} = \{ \frac{a}{b} \mid \ a \in \mathbb{Z}_{+}^{*},\ b \in \mathbb{Z}_{+}^{*} \}\)

  \(\mathbb{Q}_{+}^{*} = \{ \frac{a}{b} \mid \ a \in \mathbb{Z}_{-}^{*},\ b \in \mathbb{Z}_{-}^{*} \}\)
\item
  Números racionais negativos

  \(\mathbb{Q}_{-}^{*} = \{ \frac{a}{b} \mid \ a \in \mathbb{Z}_{-}^{*},\ b \in \mathbb{Z}_{+}^{*} \}\)

  \(\mathbb{Q}_{-}^{*} = \{ \frac{a}{b} \mid \ a \in \mathbb{Z}_{+}^{*},\ b \in \mathbb{Z}_{-}^{*} \}\)
\end{enumerate}

\section{Números Irracionais}\label{numeros-irracionais}

\section{Números Reais}\label{numeros-reais}

\section{Operações com Conjuntos}\label{operacoes-com-conjuntos}

\chapter{Sem Título}\label{sem-titulo}

\chapter{Sem Título}\label{sem-titulo-1}

\chapter{Sem Título}\label{sem-titulo-2}

\chapter{Sem Título}\label{sem-titulo-3}

\bibliography{book.bib,packages.bib}


\end{document}
